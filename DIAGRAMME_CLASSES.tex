\documentclass[12pt,a4paper]{article}
\usepackage[utf8]{inputenc}
\usepackage[french]{babel}
\usepackage{geometry}
\usepackage{graphicx}
\usepackage{hyperref}
\usepackage{listings}
\usepackage{xcolor}
\usepackage{enumitem}
\usepackage{amsmath}
\usepackage{amsfonts}
\usepackage{booktabs}
\usepackage{array}
\usepackage{tikz}
\usetikzlibrary{shapes,arrows,positioning,shadows}

\geometry{margin=2.5cm}

\lstset{
    language=Java,
    basicstyle=\ttfamily\small,
    backgroundcolor=\color{gray!10},
    frame=single,
    breaklines=true,
    showstringspaces=false
}

\title{\textbf{Diagramme de Classes - Projet COSONE}}
\author{Étudiant en Stage}
\date{\today}

\begin{document}

\maketitle

\section{Vue d'ensemble}

Ce diagramme présente l'architecture des classes du système COSONE, montrant les entités, contrôleurs, services et repositories avec leurs relations.

\section{Diagramme de Classes UML}

\begin{figure}[h]
\centering
\begin{tikzpicture}[node distance=1.5cm, auto]
    % Style des nœuds
    \tikzstyle{class} = [rectangle, draw, fill=blue!20, text width=4cm, text centered, minimum height=1.5cm, drop shadow]
    \tikzstyle{enum} = [rectangle, draw, fill=green!20, text width=3cm, text centered, minimum height=1cm, drop shadow]
    \tikzstyle{controller} = [rectangle, draw, fill=orange!20, text width=3.5cm, text centered, minimum height=1.2cm, drop shadow]
    \tikzstyle{service} = [rectangle, draw, fill=yellow!20, text width=3cm, text centered, minimum height=1cm, drop shadow]
    \tikzstyle{repository} = [rectangle, draw, fill=purple!20, text width=3cm, text centered, minimum height=1cm, drop shadow]
    \tikzstyle{arrow} = [thick,->,>=stealth]
    \tikzstyle{relation} = [thick,<->,>=stealth]
    
    % Entités principales
    \node (user) [class] {\textbf{User}\\\textit{-id: Long}\\\textit{-username: String}\\\textit{-password: String}\\\textit{-numCin: String}\\\textit{-matricule: String}\\\textit{-phoneNumber: String}\\\textit{-role: String}};
    
    \node (reservation) [class, right of=user, xshift=4cm] {\textbf{Reservation}\\\textit{-id: Long}\\\textit{-matricule: String}\\\textit{-cin: String}\\\textit{-telephone: String}\\\textit{-email: String}\\\textit{-dateDebut: LocalDateTime}\\\textit{-dateFin: LocalDateTime}\\\textit{-centre: Centre}\\\textit{-typeLogement: TypeLogement}\\\textit{-nombrePersonnes: Integer}\\\textit{-statut: StatutReservation}};
    
    \node (centre) [class, below of=user, yshift=-2cm] {\textbf{Centre}\\\textit{-id: Long}\\\textit{-nom: String}\\\textit{-adresse: String}\\\textit{-ville: String}\\\textit{-telephone: String}\\\textit{-email: String}\\\textit{-description: String}\\\textit{-actif: Boolean}};
    
    \node (typeLogement) [class, right of=centre, xshift=4cm] {\textbf{TypeLogement}\\\textit{-id: Long}\\\textit{-nom: String}\\\textit{-description: String}\\\textit{-capaciteMax: Integer}\\\textit{-prixParNuit: Double}\\\textit{-actif: Boolean}};
    
    % Enums
    \node (statut) [enum, below of=reservation, yshift=-2cm] {\textbf{StatutReservation}\\\textit{EN\_ATTENTE\_PAIEMENT}\\\textit{PAYEE}\\\textit{CONFIRMEE}\\\textit{ANNULEE}\\\textit{EXPIREE}};
    
    \node (methode) [enum, right of=statut, xshift=3cm] {\textbf{MethodePaiement}\\\textit{CARTE\_BANCAIRE}\\\textit{VIREMENT}\\\textit{ESPECES}\\\textit{CHEQUE}\\\textit{MOBILE\_MONEY}\\\textit{AUTRE}};
    
    % Contrôleurs
    \node (homeController) [controller, below of=centre, yshift=-3cm] {\textbf{HomeController}\\\textit{-homeContentService}\\\textit{-centresCsvService}\\\textit{+root(): String}\\\textit{+home(): String}\\\textit{+landing(): String}};
    
    \node (authController) [controller, right of=homeController, xshift=4cm] {\textbf{AuthController}\\\textit{-userRepository}\\\textit{-externAuthCodeRepository}\\\textit{-smsService}\\\textit{+showLogin(): String}\\\textit{+register(): String}\\\textit{+verifyPhone(): String}};
    
    \node (reservationController) [controller, below of=homeController, yshift=-2cm] {\textbf{ReservationController}\\\textit{-reservationService}\\\textit{-centreRepository}\\\textit{-typeLogementRepository}\\\textit{+afficherPageReservation(): String}\\\textit{+creerReservation(): String}\\\textit{+confirmerPaiement(): String}};
    
    % Services
    \node (reservationService) [service, right of=reservationController, xshift=4cm] {\textbf{ReservationService}\\\textit{-reservationRepository}\\\textit{-centreRepository}\\\textit{+creerReservation(): Reservation}\\\textit{+confirmerPaiement(): Reservation}\\\textit{+estDisponible(): boolean}};
    
    \node (emailService) [service, below of=reservationService, yshift=-1.5cm] {\textbf{EmailService}\\\textit{+envoyerEmail(): void}\\\textit{+envoyerConfirmationReservation(): void}};
    
    \node (smsService) [service, left of=emailService, xshift=-3cm] {\textbf{SmsService}\\\textit{+envoyerSms(): void}\\\textit{+genererCodeVerification(): String}};
    
    % Repositories
    \node (userRepository) [repository, below of=homeController, yshift=-4cm] {\textbf{UserRepository}\\\textit{+findByUsername(): Optional<User>}\\\textit{+findByMatricule(): Optional<User>}\\\textit{+findByPhoneNumber(): Optional<User>}};
    
    \node (reservationRepository) [repository, right of=userRepository, xshift=3cm] {\textbf{ReservationRepository}\\\textit{+findByMatricule(): List<Reservation>}\\\textit{+findByStatut(): List<Reservation>}\\\textit{+findByDateDebutBetween(): List<Reservation>}};
    
    \node (centreRepository) [repository, right of=reservationRepository, xshift=3cm] {\textbf{CentreRepository}\\\textit{+findByActifTrueOrderByNom(): List<Centre>}\\\textit{+findByVille(): List<Centre>}};
    
    % Relations
    \draw [relation] (reservation) -- (centre);
    \draw [relation] (reservation) -- (typeLogement);
    \draw [relation] (reservation) -- (statut);
    \draw [relation] (reservation) -- (methode);
    
    \draw [arrow] (homeController) -- (userRepository);
    \draw [arrow] (authController) -- (userRepository);
    \draw [arrow] (reservationController) -- (reservationService);
    \draw [arrow] (reservationService) -- (reservationRepository);
    \draw [arrow] (reservationService) -- (centreRepository);
    
\end{tikzpicture}
\caption{Diagramme de classes UML du système COSONE}
\end{figure}

\section{Description des Classes}

\subsection{Entités (Model)}

\subsubsection{User}
Représente un utilisateur du système avec ses informations d'authentification.

\begin{lstlisting}[caption=Classe User]
@Entity
@Table(name = "users")
public class User {
    @Id
    @GeneratedValue(strategy = GenerationType.IDENTITY)
    private Long id;
    
    @Column(nullable = false, unique = true)
    private String username;
    
    @Column(nullable = true)
    private String password;
    
    @Column(nullable = false, unique = true)
    private String numCin;
    
    @Column(nullable = false, unique = true)
    private String matricule;
    
    @Column(nullable = false, unique = true)
    private String phoneNumber;
    
    @Column(nullable = false)
    private String role = "USER";
}
\end{lstlisting}

\subsubsection{Reservation}
Entité centrale pour les réservations avec logique métier intégrée.

\begin{lstlisting}[caption=Classe Reservation - Méthodes principales]
public class Reservation {
    // ... attributs ...
    
    public boolean isEnRetardPaiement() {
        return LocalDateTime.now().isAfter(dateLimitePaiement) 
               && statut == StatutReservation.EN_ATTENTE_PAIEMENT;
    }
    
    public boolean peutEtreReservee() {
        return dateDebut.isAfter(LocalDateTime.now().plusDays(10));
    }
    
    public boolean estWeekend() {
        int dayOfWeek = dateDebut.getDayOfWeek().getValue();
        return dayOfWeek == 6 || dayOfWeek == 7;
    }
}
\end{lstlisting}

\subsubsection{Centre}
Informations sur les centres de vacances.

\subsubsection{TypeLogement}
Types de logements disponibles avec tarification.

\subsubsection{PersonneAccompagnement}
Personnes accompagnant les réservations.

\subsection{Enums}

\subsubsection{StatutReservation}
Statuts possibles d'une réservation.

\begin{lstlisting}[caption=Enum StatutReservation]
public enum StatutReservation {
    EN_ATTENTE_PAIEMENT("En attente de paiement"),
    PAYEE("Payée"),
    CONFIRMEE("Confirmée"),
    ANNULEE("Annulée"),
    EXPIREE("Expirée");
    
    private final String libelle;
    
    public String getLibelle() {
        return libelle;
    }
}
\end{lstlisting}

\subsubsection{MethodePaiement}
Méthodes de paiement acceptées.

\subsection{Contrôleurs (Controller)}

\subsubsection{HomeController}
Gestion de la page d'accueil et landing page.

\subsubsection{AuthController}
Authentification, inscription et vérification.

\subsubsection{ReservationController}
Gestion complète des réservations.

\subsubsection{EspaceReservationController}
Espace personnel des utilisateurs.

\subsection{Services}

\subsubsection{ReservationService}
Logique métier des réservations.

\begin{lstlisting}[caption=Service ReservationService - Méthodes principales]
@Service
public class ReservationService {
    @Autowired
    private ReservationRepository reservationRepository;
    
    public Reservation creerReservation(Reservation reservation) {
        // Logique de création
    }
    
    public boolean estDisponible(Reservation reservation) {
        // Vérification de disponibilité
    }
    
    public Double calculerPrixTotal(Reservation reservation) {
        // Calcul du prix
    }
}
\end{lstlisting}

\subsubsection{EmailService}
Envoi d'emails.

\subsubsection{SmsService}
Envoi de SMS.

\subsubsection{WordPressService}
Intégration WordPress.

\subsection{Repositories}

Interfaces JPA pour l'accès aux données avec méthodes de requête personnalisées.

\begin{lstlisting}[caption=Repository UserRepository]
@Repository
public interface UserRepository extends JpaRepository<User, Long> {
    Optional<User> findByUsername(String username);
    Optional<User> findByMatricule(String matricule);
    Optional<User> findByPhoneNumber(String phoneNumber);
    Optional<User> findByNumCin(String numCin);
}
\end{lstlisting}

\section{Relations entre Classes}

\subsection{Relations d'Association}

\begin{itemize}
    \item \textbf{Reservation} $\rightarrow$ \textbf{Centre} (Many-to-One)
    \item \textbf{Reservation} $\rightarrow$ \textbf{TypeLogement} (Many-to-One)
    \item \textbf{Reservation} $\rightarrow$ \textbf{PersonneAccompagnement} (One-to-Many)
    \item \textbf{Reservation} $\rightarrow$ \textbf{StatutReservation} (One-to-One)
    \item \textbf{Reservation} $\rightarrow$ \textbf{MethodePaiement} (One-to-One)
\end{itemize}

\subsection{Relations de Dépendance}

\begin{itemize}
    \item \textbf{Controllers} $\rightarrow$ \textbf{Services}
    \item \textbf{Services} $\rightarrow$ \textbf{Repositories}
    \item \textbf{Repositories} $\rightarrow$ \textbf{Entities}
\end{itemize}

\section{Principes de Design}

\subsection{Séparation des Responsabilités}
Chaque classe a une responsabilité claire et bien définie :
\begin{itemize}
    \item \textbf{Entities} : Représentation des données
    \item \textbf{Controllers} : Gestion des requêtes HTTP
    \item \textbf{Services} : Logique métier
    \item \textbf{Repositories} : Accès aux données
\end{itemize}

\subsection{Inversion de Dépendance}
Les contrôleurs dépendent des services, pas des repositories directement.

\subsection{Encapsulation}
Données privées avec accesseurs publics pour contrôler l'accès.

\subsection{Cohésion}
Classes fortement cohésives avec des méthodes liées.

\subsection{Couplage Faible}
Dépendances minimisées entre les classes grâce à l'injection de dépendances Spring.

\section{Patterns Utilisés}

\subsection{Model-View-Controller (MVC)}
Architecture claire séparant la logique métier, la présentation et le contrôle.

\subsection{Repository Pattern}
Abstraction de l'accès aux données.

\subsection{Dependency Injection}
Injection automatique des dépendances par Spring.

\subsection{Service Layer Pattern}
Couche de service pour la logique métier.

Cette architecture respecte les bonnes pratiques de développement et facilite la maintenance et l'évolution du système.

\end{document}
