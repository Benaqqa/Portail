\documentclass[12pt,a4paper]{article}
\usepackage[utf8]{inputenc}
\usepackage[french]{babel}
\usepackage{geometry}
\usepackage{graphicx}
\usepackage{hyperref}
\usepackage{listings}
\usepackage{xcolor}
\usepackage{enumitem}
\usepackage{amsmath}
\usepackage{amsfonts}
\usepackage{booktabs}
\usepackage{array}
\usepackage{tikz}
\usetikzlibrary{shapes,arrows,positioning}

\geometry{margin=2.5cm}

\lstset{
    language=SQL,
    basicstyle=\ttfamily\small,
    backgroundcolor=\color{gray!10},
    frame=single,
    breaklines=true,
    showstringspaces=false
}

\title{\textbf{Schéma de Base de Données - Projet COSONE}}
\author{Étudiant en Stage}
\date{\today}

\begin{document}

\maketitle

\section{Vue d'ensemble}

Le projet COSONE utilise une base de données \textbf{PostgreSQL} avec une architecture relationnelle basée sur les entités JPA. Le schéma comprend 7 tables principales pour la gestion des utilisateurs, réservations, centres et authentification.

\section{Diagramme de Base de Données}

\begin{figure}[h]
\centering
\begin{tikzpicture}[node distance=3.5cm, auto, scale=0.5, transform shape]
    % Style des nœuds
    \tikzstyle{table} = [rectangle, draw, fill=blue!20, text width=4cm, text centered, minimum height=1cm, drop shadow]
    \tikzstyle{arrow} = [thick,->,>=stealth]
    
    % Tables principales - Ligne du haut (centrées)
    \node (users) [table, xshift=-4.5cm] {\textbf{USERS}};
    \node (centres) [table, right of=users, xshift=3cm] {\textbf{CENTRES}};
    \node (types) [table, right of=centres, xshift=3cm] {\textbf{TYPES\_LOGEMENT}};
    
    % Tables de réservation - Ligne du milieu (centrées)
    \node (reservations) [table, below of=users, yshift=-1.8cm, xshift=-4.5cm] {\textbf{RESERVATIONS}};
    \node (personnes) [table, right of=reservations, xshift=3cm] {\textbf{PERSONNES\_ACCOMP.}};
    \node (extern) [table, right of=personnes, xshift=3cm] {\textbf{EXTERN\_AUTH\_CODES}};
    
    % Table de vérification - Ligne du bas (centrée)
    \node (phone) [table, below of=reservations, yshift=-1.8cm, xshift=-4.5cm] {\textbf{PHONE\_VERIF\_CODES}};
    
    % Relations principales
    \draw [arrow] (reservations) -- node[above, sloped] {centre\_id} (centres);
    \draw [arrow] (reservations) -- node[above, sloped] {type\_logement\_id} (types);
    \draw [arrow] (personnes) -- node[above, sloped] {reservation\_id} (reservations);
    
    % Légende
    \node [below of=phone, yshift=-1cm, xshift=-4.5cm, text width=12cm, text centered] {
        \textbf{Relations :} Reservations $\rightarrow$ Centres, Reservations $\rightarrow$ Types de Logement, Personnes $\rightarrow$ Reservations
    };
\end{tikzpicture}
\caption{Diagramme de base de données COSONE}
\end{figure}

\section{Description Détaillée des Tables}

\subsection{Table \texttt{users}}
\textbf{Description} : Stocke les informations des utilisateurs du système

\begin{table}[h]
\centering
\begin{tabular}{|l|l|l|p{6cm}|}
\hline
\textbf{Colonne} & \textbf{Type} & \textbf{Contraintes} & \textbf{Description} \\
\hline
id & BIGSERIAL & PRIMARY KEY & Identifiant unique \\
\hline
username & VARCHAR & NOT NULL, UNIQUE & Nom d'utilisateur \\
\hline
password & VARCHAR & NULL & Mot de passe (peut être null) \\
\hline
num\_cin & VARCHAR & NOT NULL, UNIQUE & Numéro CIN \\
\hline
matricule & VARCHAR & NOT NULL, UNIQUE & Matricule employé \\
\hline
phone\_number & VARCHAR & NOT NULL, UNIQUE & Numéro de téléphone \\
\hline
role & VARCHAR & NOT NULL, DEFAULT 'USER' & Rôle utilisateur (USER/ADMIN) \\
\hline
\end{tabular}
\caption{Structure de la table users}
\end{table}

\subsection{Table \texttt{centres}}
\textbf{Description} : Informations sur les centres de vacances

\begin{table}[h]
\centering
\begin{tabular}{|l|l|l|p{6cm}|}
\hline
\textbf{Colonne} & \textbf{Type} & \textbf{Contraintes} & \textbf{Description} \\
\hline
id & BIGSERIAL & PRIMARY KEY & Identifiant unique \\
\hline
nom & VARCHAR & NOT NULL, UNIQUE & Nom du centre \\
\hline
adresse & VARCHAR & NOT NULL & Adresse du centre \\
\hline
ville & VARCHAR & NOT NULL & Ville \\
\hline
telephone & VARCHAR & NOT NULL & Téléphone du centre \\
\hline
email & VARCHAR & NOT NULL & Email du centre \\
\hline
description & TEXT & NULL & Description du centre \\
\hline
actif & BOOLEAN & NOT NULL, DEFAULT true & Centre actif/inactif \\
\hline
\end{tabular}
\caption{Structure de la table centres}
\end{table}

\subsection{Table \texttt{types\_logement}}
\textbf{Description} : Types de logements disponibles

\begin{table}[h]
\centering
\begin{tabular}{|l|l|l|p{6cm}|}
\hline
\textbf{Colonne} & \textbf{Type} & \textbf{Contraintes} & \textbf{Description} \\
\hline
id & BIGSERIAL & PRIMARY KEY & Identifiant unique \\
\hline
nom & VARCHAR & NOT NULL, UNIQUE & Nom du type \\
\hline
description & VARCHAR & NOT NULL & Description \\
\hline
capacite\_max & INTEGER & NOT NULL & Capacité maximale \\
\hline
prix\_par\_nuit & DECIMAL & NOT NULL & Prix par nuit \\
\hline
actif & BOOLEAN & NOT NULL, DEFAULT true & Type actif/inactif \\
\hline
\end{tabular}
\caption{Structure de la table types\_logement}
\end{table}

\subsection{Table \texttt{reservations}}
\textbf{Description} : Réservations des utilisateurs

\begin{table}[h]
\centering
\begin{tabular}{|l|l|l|p{6cm}|}
\hline
\textbf{Colonne} & \textbf{Type} & \textbf{Contraintes} & \textbf{Description} \\
\hline
id & BIGSERIAL & PRIMARY KEY & Identifiant unique \\
\hline
matricule & VARCHAR & NOT NULL & Matricule du réservant \\
\hline
cin & VARCHAR & NOT NULL & CIN du réservant \\
\hline
telephone & VARCHAR & NOT NULL & Téléphone du réservant \\
\hline
email & VARCHAR & NOT NULL & Email du réservant \\
\hline
date\_debut & TIMESTAMP & NOT NULL & Date de début \\
\hline
date\_fin & TIMESTAMP & NOT NULL & Date de fin \\
\hline
centre\_id & BIGINT & NOT NULL, FK & Référence au centre \\
\hline
type\_logement\_id & BIGINT & NOT NULL, FK & Référence au type de logement \\
\hline
nombre\_personnes & INTEGER & NOT NULL & Nombre de personnes \\
\hline
statut & VARCHAR & NOT NULL & Statut de la réservation \\
\hline
date\_reservation & TIMESTAMP & NOT NULL & Date de création \\
\hline
date\_limite\_paiement & TIMESTAMP & NOT NULL & Date limite de paiement \\
\hline
date\_paiement & TIMESTAMP & NULL & Date de paiement \\
\hline
methode\_paiement & VARCHAR & NULL & Méthode de paiement \\
\hline
reference\_paiement & VARCHAR & NULL & Référence de paiement \\
\hline
commentaires & TEXT & NULL & Commentaires \\
\hline
\end{tabular}
\caption{Structure de la table reservations}
\end{table}

\subsection{Table \texttt{personnes\_accompagnement}}
\textbf{Description} : Personnes accompagnant les réservations

\begin{table}[h]
\centering
\begin{tabular}{|l|l|l|p{6cm}|}
\hline
\textbf{Colonne} & \textbf{Type} & \textbf{Contraintes} & \textbf{Description} \\
\hline
id & BIGSERIAL & PRIMARY KEY & Identifiant unique \\
\hline
reservation\_id & BIGINT & NOT NULL, FK & Référence à la réservation \\
\hline
nom & VARCHAR & NOT NULL & Nom de famille \\
\hline
prenom & VARCHAR & NOT NULL & Prénom \\
\hline
cin & VARCHAR & NOT NULL & Numéro CIN \\
\hline
lien\_parente & VARCHAR & NOT NULL & Lien de parenté \\
\hline
\end{tabular}
\caption{Structure de la table personnes\_accompagnement}
\end{table}

\subsection{Table \texttt{extern\_auth\_codes}}
\textbf{Description} : Codes d'authentification externe

\begin{table}[h]
\centering
\begin{tabular}{|l|l|l|p{6cm}|}
\hline
\textbf{Colonne} & \textbf{Type} & \textbf{Contraintes} & \textbf{Description} \\
\hline
id & BIGSERIAL & PRIMARY KEY & Identifiant unique \\
\hline
code & VARCHAR & NOT NULL, UNIQUE & Code d'authentification \\
\hline
used & BOOLEAN & NOT NULL, DEFAULT false & Code utilisé ou non \\
\hline
created\_at & TIMESTAMP & NOT NULL & Date de création \\
\hline
prenom & VARCHAR & NOT NULL & Prénom associé \\
\hline
nom & VARCHAR & NOT NULL & Nom associé \\
\hline
\end{tabular}
\caption{Structure de la table extern\_auth\_codes}
\end{table}

\subsection{Table \texttt{phone\_verification\_codes}}
\textbf{Description} : Codes de vérification SMS

\begin{table}[h]
\centering
\begin{tabular}{|l|l|l|p{6cm}|}
\hline
\textbf{Colonne} & \textbf{Type} & \textbf{Contraintes} & \textbf{Description} \\
\hline
id & BIGSERIAL & PRIMARY KEY & Identifiant unique \\
\hline
phone\_number & VARCHAR & NOT NULL & Numéro de téléphone \\
\hline
code & VARCHAR & NOT NULL & Code de vérification \\
\hline
used & BOOLEAN & NOT NULL, DEFAULT false & Code utilisé ou non \\
\hline
created\_at & TIMESTAMP & NOT NULL & Date de création \\
\hline
expires\_at & TIMESTAMP & NOT NULL & Date d'expiration \\
\hline
\end{tabular}
\caption{Structure de la table phone\_verification\_codes}
\end{table}

\section{Relations entre Tables}

\subsection{Relations Principales}

\begin{enumerate}
    \item \textbf{Reservations $\rightarrow$ Centres} (Many-to-One)
    \begin{itemize}
        \item \texttt{reservations.centre\_id} $\rightarrow$ \texttt{centres.id}
        \item Une réservation appartient à un centre
    \end{itemize}
    
    \item \textbf{Reservations $\rightarrow$ Types de Logement} (Many-to-One)
    \begin{itemize}
        \item \texttt{reservations.type\_logement\_id} $\rightarrow$ \texttt{types\_logement.id}
        \item Une réservation concerne un type de logement
    \end{itemize}
    
    \item \textbf{Personnes Accompagnement $\rightarrow$ Reservations} (One-to-Many)
    \begin{itemize}
        \item \texttt{personnes\_accompagnement.reservation\_id} $\rightarrow$ \texttt{reservations.id}
        \item Une réservation peut avoir plusieurs personnes d'accompagnement
    \end{itemize}
\end{enumerate}

\subsection{Contraintes d'Intégrité}

\begin{itemize}
    \item \textbf{Clés étrangères} : Toutes les références sont validées
    \item \textbf{Unicité} : Username, CIN, matricule, numéro de téléphone uniques
    \item \textbf{Non-nullité} : Champs obligatoires définis
    \item \textbf{Cohérence temporelle} : Dates de début < dates de fin
\end{itemize}

\section{Index Recommandés}

\begin{lstlisting}[caption=Index pour les recherches fréquentes]
-- Index pour les recherches fréquentes
CREATE INDEX idx_reservations_matricule ON reservations(matricule);
CREATE INDEX idx_reservations_dates ON reservations(date_debut, date_fin);
CREATE INDEX idx_reservations_statut ON reservations(statut);
CREATE INDEX idx_users_matricule ON users(matricule);
CREATE INDEX idx_phone_codes_expires ON phone_verification_codes(expires_at);
\end{lstlisting}

\section{Scripts de Création}

Le schéma est automatiquement généré par Hibernate avec la configuration :
\begin{lstlisting}[caption=Configuration Hibernate]
spring.jpa.hibernate.ddl-auto=update
\end{lstlisting}

Cette configuration permet la mise à jour automatique du schéma lors des modifications des entités JPA.

\section{Sécurité et Performance}

\begin{itemize}
    \item \textbf{Chiffrement} : Les mots de passe sont chiffrés par Spring Security
    \item \textbf{Validation} : Contraintes de validation au niveau base et application
    \item \textbf{Performance} : Index sur les colonnes fréquemment utilisées
    \item \textbf{Sauvegarde} : Recommandation de sauvegardes régulières PostgreSQL
\end{itemize}

\end{document}
