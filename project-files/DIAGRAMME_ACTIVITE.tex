\documentclass[12pt,a4paper]{article}
\usepackage[utf8]{inputenc}
\usepackage[french]{babel}
\usepackage{geometry}
\usepackage{graphicx}
\usepackage{hyperref}
\usepackage{listings}
\usepackage{xcolor}
\usepackage{enumitem}
\usepackage{amsmath}
\usepackage{amsfonts}
\usepackage{booktabs}
\usepackage{array}
\usepackage{tikz}
\usetikzlibrary{shapes,arrows,positioning,shadows,decorations.pathmorphing}

\geometry{margin=2.5cm}

\title{\textbf{Diagramme d'Activité - Portail COS'ONE}}
\author{BENAQQA Moubarak}
\date{\today}

\begin{document}

\maketitle

\section{Vue d'ensemble}

Ce document présente les diagrammes d'activité du système COSONE, décrivant les flux de processus pour les principales fonctionnalités du système.

\section{Processus d'Inscription Utilisateur et Création de Réservation}

\begin{figure}[h]
\centering
\begin{minipage}[t]{0.48\textwidth}
\centering
\begin{tikzpicture}[node distance=3cm, auto, scale=0.35, transform shape]
    % Style des nœuds
    \tikzstyle{start} = [ellipse, draw, fill=green!20, text width=1.8cm, text centered, minimum height=0.7cm]
    \tikzstyle{end} = [ellipse, draw, fill=red!20, text width=1.8cm, text centered, minimum height=0.7cm]
    \tikzstyle{process} = [rectangle, draw, fill=blue!20, text width=2.2cm, text centered, minimum height=0.7cm]
    \tikzstyle{decision} = [diamond, draw, fill=yellow!20, text width=1.8cm, text centered, minimum height=0.7cm]
    \tikzstyle{arrow} = [thick,->,>=stealth]
    
    % Nœuds
    \node (start) [start] {Début};
    \node (a) [process, below of=start, yshift=-0.4cm] {Utilisateur accède à\\la page d'inscription};
    \node (b) [process, below of=a, yshift=-0.4cm] {Utilisateur saisit le\\code d'authentification};
    \node (c) [decision, below of=b, yshift=-0.4cm] {Code valide ?};
    \node (d) [process, left of=c, xshift=-3cm] {Afficher erreur};
    \node (e) [process, below of=c, yshift=-0.4cm] {Utilisateur saisit ses\\informations personnelles};
    \node (f) [process, below of=e, yshift=-0.4cm] {Vérifier l'unicité\\des données};
    \node (g) [decision, below of=f, yshift=-0.4cm] {Données uniques ?};
    \node (h) [process, left of=g, xshift=-3cm] {Afficher erreur\\de doublon};
    \node (i) [process, below of=g, yshift=-0.4cm] {Créer le compte\\utilisateur};
    \node (j) [process, below of=i, yshift=-0.4cm] {Générer un code SMS\\de vérification};
    \node (k) [process, below of=j, yshift=-0.4cm] {Envoyer SMS\\avec le code};
    \node (l) [process, below of=k, yshift=-0.4cm] {Rediriger vers la page\\de vérification SMS};
    \node (m) [process, below of=l, yshift=-0.4cm] {Utilisateur saisit\\le code SMS reçu};
    \node (n) [decision, below of=m, yshift=-0.4cm] {Code SMS valide ?};
    \node (o) [process, left of=n, xshift=-3cm] {Afficher erreur};
    \node (p) [process, below of=n, yshift=-0.4cm] {Marquer le numéro\\comme vérifié};
    \node (q) [process, below of=p, yshift=-0.4cm] {Rediriger vers la création\\de mot de passe};
    \node (r) [process, below of=q, yshift=-0.4cm] {Utilisateur crée\\son mot de passe};
    \node (s) [process, below of=r, yshift=-0.4cm] {Valider la force\\du mot de passe};
    \node (t) [decision, below of=s, yshift=-0.4cm] {Mot de passe valide ?};
    \node (u) [process, left of=t, xshift=-3cm] {Afficher erreur\\de validation};
    \node (v) [process, below of=t, yshift=-0.4cm] {Enregistrer le\\mot de passe};
    \node (w) [process, below of=v, yshift=-0.4cm] {Inscription terminée\\avec succès};
    \node (end) [end, below of=w, yshift=-0.4cm] {Fin};
    
    % Flèches
    \draw [arrow] (start) -- (a);
    \draw [arrow] (a) -- (b);
    \draw [arrow] (b) -- (c);
    \draw [arrow] (c) -- node[above, sloped] {Non} (d);
    \draw [arrow] (d) -- (b);
    \draw [arrow] (c) -- node[right, yshift=0.2cm] {Oui} (e);
    \draw [arrow] (e) -- (f);
    \draw [arrow] (f) -- (g);
    \draw [arrow] (g) -- node[above, sloped] {Non} (h);
    \draw [arrow] (h) -- (e);
    \draw [arrow] (g) -- node[right, yshift=0.2cm] {Oui} (i);
    \draw [arrow] (i) -- (j);
    \draw [arrow] (j) -- (k);
    \draw [arrow] (k) -- (l);
    \draw [arrow] (l) -- (m);
    \draw [arrow] (m) -- (n);
    \draw [arrow] (n) -- node[above, sloped] {Non} (o);
    \draw [arrow] (o) -- (m);
    \draw [arrow] (n) -- node[right, yshift=0.2cm] {Oui} (p);
    \draw [arrow] (p) -- (q);
    \draw [arrow] (q) -- (r);
    \draw [arrow] (r) -- (s);
    \draw [arrow] (s) -- (t);
    \draw [arrow] (t) -- node[above, sloped] {Non} (u);
    \draw [arrow] (u) -- (r);
    \draw [arrow] (t) -- node[right, yshift=0.2cm] {Oui} (v);
    \draw [arrow] (v) -- (w);
    \draw [arrow] (w) -- (end);
    
\end{tikzpicture}
\caption{Processus d'inscription utilisateur}
\end{minipage}
\hfill
\begin{minipage}[t]{0.48\textwidth}
\centering
\begin{tikzpicture}[node distance=2.8cm, auto, scale=0.35, transform shape]
    % Style des nœuds
    \tikzstyle{start} = [ellipse, draw, fill=green!20, text width=1.8cm, text centered, minimum height=0.7cm]
    \tikzstyle{end} = [ellipse, draw, fill=red!20, text width=1.8cm, text centered, minimum height=0.7cm]
    \tikzstyle{process} = [rectangle, draw, fill=blue!20, text width=2.2cm, text centered, minimum height=0.7cm]
    \tikzstyle{decision} = [diamond, draw, fill=yellow!20, text width=1.8cm, text centered, minimum height=0.7cm]
    \tikzstyle{arrow} = [thick,->,>=stealth]
    
    % Nœuds
    \node (start) [start] {Début};
    \node (a) [process, below of=start, yshift=-0.35cm] {Utilisateur connecté\\accède à la réservation};
    \node (b) [process, below of=a, yshift=-0.35cm] {Afficher la liste\\des centres disponibles};
    \node (c) [process, below of=b, yshift=-0.35cm] {Utilisateur sélectionne\\un centre};
    \node (d) [process, below of=c, yshift=-0.35cm] {Afficher les types\\de logement du centre};
    \node (e) [process, below of=d, yshift=-0.35cm] {Utilisateur sélectionne\\un type de logement};
    \node (f) [process, below of=e, yshift=-0.35cm] {Utilisateur choisit\\les dates de séjour};
    \node (g) [process, below of=f, yshift=-0.35cm] {Vérifier que les dates\\sont valides};
    \node (h) [decision, below of=g, yshift=-0.35cm] {Dates valides ?};
    \node (i) [process, left of=h, xshift=-3cm] {Afficher erreur\\de dates};
    \node (j) [process, below of=h, yshift=-0.35cm] {Vérifier la\\disponibilité};
    \node (k) [decision, below of=j, yshift=-0.35cm] {Disponible ?};
    \node (l) [process, left of=k, xshift=-3cm] {Afficher message\\d'indisponibilité};
    \node (m) [process, below of=k, yshift=-0.35cm] {Utilisateur saisit\\le nombre de personnes};
    \node (n) [process, below of=m, yshift=-0.35cm] {Vérifier la capacité\\du logement};
    \node (o) [decision, below of=n, yshift=-0.35cm] {Capacité suffisante ?};
    \node (p) [process, left of=o, xshift=-3cm] {Afficher erreur\\de capacité};
    \node (q) [process, below of=o, yshift=-0.35cm] {Calculer le prix total};
    \node (r) [process, below of=q, yshift=-0.35cm] {Afficher le récapitulatif\\et le prix};
    \node (s) [process, below of=r, yshift=-0.35cm] {Utilisateur ajoute les\\personnes d'accompagnement};
    \node (t) [process, below of=s, yshift=-0.35cm] {Utilisateur confirme\\la réservation};
    \node (u) [process, below of=t, yshift=-0.35cm] {Créer la réservation\\en base};
    \node (v) [process, below of=u, yshift=-0.35cm] {Générer la date limite\\de paiement};
    \node (w) [process, below of=v, yshift=-0.35cm] {Envoyer email\\de confirmation};
    \node (x) [process, below of=w, yshift=-0.35cm] {Afficher la page\\de confirmation};
    \node (end) [end, below of=x, yshift=-0.35cm] {Fin};
    
    % Flèches
    \draw [arrow] (start) -- (a);
    \draw [arrow] (a) -- (b);
    \draw [arrow] (b) -- (c);
    \draw [arrow] (c) -- (d);
    \draw [arrow] (d) -- (e);
    \draw [arrow] (e) -- (f);
    \draw [arrow] (f) -- (g);
    \draw [arrow] (g) -- (h);
    \draw [arrow] (h) -- node[above, sloped] {Non} (i);
    \draw [arrow] (i) -- (f);
    \draw [arrow] (h) -- node[right, yshift=0.2cm] {Oui} (j);
    \draw [arrow] (j) -- (k);
    \draw [arrow] (k) -- node[above, sloped] {Non} (l);
    \draw [arrow] (l) -- (f);
    \draw [arrow] (k) -- node[right, yshift=0.2cm] {Oui} (m);
    \draw [arrow] (m) -- (n);
    \draw [arrow] (n) -- (o);
    \draw [arrow] (o) -- node[above, sloped] {Non} (p);
    \draw [arrow] (p) -- (m);
    \draw [arrow] (o) -- node[right, yshift=0.2cm] {Oui} (q);
    \draw [arrow] (q) -- (r);
    \draw [arrow] (r) -- (s);
    \draw [arrow] (s) -- (t);
    \draw [arrow] (t) -- (u);
    \draw [arrow] (u) -- (v);
    \draw [arrow] (v) -- (w);
    \draw [arrow] (w) -- (x);
    \draw [arrow] (x) -- (end);
    
\end{tikzpicture}
\caption{Processus de création de réservation}
\end{minipage}
\end{figure}

\section{Processus de Paiement et Vérification de Disponibilité}

\begin{figure}[h]
\centering
\begin{minipage}[t]{0.48\textwidth}
\centering
\begin{tikzpicture}[node distance=3.5cm, auto, scale=0.35, transform shape]
    % Style des nœuds
    \tikzstyle{start} = [ellipse, draw, fill=green!20, text width=1.8cm, text centered, minimum height=0.7cm]
    \tikzstyle{end} = [ellipse, draw, fill=red!20, text width=1.8cm, text centered, minimum height=0.7cm]
    \tikzstyle{process} = [rectangle, draw, fill=blue!20, text width=2.2cm, text centered, minimum height=0.7cm]
    \tikzstyle{decision} = [diamond, draw, fill=yellow!20, text width=1.8cm, text centered, minimum height=0.7cm]
    \tikzstyle{arrow} = [thick,->,>=stealth]
    
    % Nœuds - Colonne principale
    \node (start) [start] {Début};
    \node (a) [process, below of=start, yshift=-0.5cm] {Utilisateur accède\\à sa réservation};
    \node (b) [process, below of=a, yshift=-0.5cm] {Vérifier le statut\\de la réservation};
    \node (c) [decision, below of=b, yshift=-0.5cm] {\small Statut =\\EN\_ATTENTE\_PAIEMENT ?};
    \node (e) [process, below of=c, yshift=-0.5cm] {Vérifier la date limite\\de paiement};
    \node (f) [decision, below of=e, yshift=-0.5cm] {Date limite\\dépassée ?};
    \node (i) [process, below of=f, yshift=-0.5cm] {Afficher les méthodes\\de paiement disponibles};
    \node (j) [process, below of=i, yshift=-0.5cm] {Utilisateur sélectionne\\une méthode};
    \node (k) [process, below of=j, yshift=-0.5cm] {Utilisateur saisit\\la référence de paiement};
    \node (l) [process, below of=k, yshift=-0.5cm] {Valider la référence\\de paiement};
    \node (m) [decision, below of=l, yshift=-0.5cm] {Référence valide ?};
    \node (o) [process, below of=m, yshift=-0.5cm] {Enregistrer les informations\\de paiement};
    \node (p) [process, below of=o, yshift=-0.5cm] {Marquer la réservation\\comme payée};
    \node (q) [process, below of=p, yshift=-0.5cm] {Envoyer email de\\confirmation de paiement};
    \node (r) [process, below of=q, yshift=-0.5cm] {Afficher confirmation\\de paiement};
    \node (end) [end, below of=r, yshift=-0.5cm] {Fin};
    
    % Nœuds - Branches latérales
    \node (d) [process, left of=c, xshift=-3cm] {Afficher message\\d'erreur};
    \node (g) [process, left of=f, xshift=-3cm] {Marquer la réservation\\comme expirée};
    \node (h) [process, below of=g] {Afficher message\\d'expiration};
    \node (n) [process, left of=m, xshift=-3cm] {Afficher erreur\\de référence};
    
    % Flèches principales
    \draw [arrow] (start) -- (a);
    \draw [arrow] (a) -- (b);
    \draw [arrow] (b) -- (c);
    \draw [arrow] (c) -- node[above, sloped] {Non} (d);
    \draw [arrow] (d) -- (end);
    \draw [arrow] (c) -- node[right, yshift=0.2cm] {Oui} (e);
    \draw [arrow] (e) -- (f);
    \draw [arrow] (f) -- node[above, sloped] {Oui} (g);
    \draw [arrow] (g) -- (h);
    \draw [arrow] (h) -- (end);
    \draw [arrow] (f) -- node[right, yshift=0.2cm] {Non} (i);
    \draw [arrow] (i) -- (j);
    \draw [arrow] (j) -- (k);
    \draw [arrow] (k) -- (l);
    \draw [arrow] (l) -- (m);
    \draw [arrow] (m) -- node[above, sloped] {Non} (n);
    \draw [arrow] (n) -- (k);
    \draw [arrow] (m) -- node[right, yshift=0.2cm] {Oui} (o);
    \draw [arrow] (o) -- (p);
    \draw [arrow] (p) -- (q);
    \draw [arrow] (q) -- (r);
    \draw [arrow] (r) -- (end);
    
\end{tikzpicture}
\caption{Processus de paiement}
\end{minipage}
\hfill
\begin{minipage}[t]{0.48\textwidth}
\centering
\begin{tikzpicture}[node distance=3.5cm, auto, scale=0.35, transform shape]
    % Style des nœuds
    \tikzstyle{start} = [ellipse, draw, fill=green!20, text width=1.8cm, text centered, minimum height=0.7cm]
    \tikzstyle{end} = [ellipse, draw, fill=red!20, text width=1.8cm, text centered, minimum height=0.7cm]
    \tikzstyle{process} = [rectangle, draw, fill=blue!20, text width=2.2cm, text centered, minimum height=0.7cm]
    \tikzstyle{decision} = [diamond, draw, fill=yellow!20, text width=1.8cm, text centered, minimum height=0.7cm]
    \tikzstyle{arrow} = [thick,->,>=stealth]
    
    % Nœuds - Colonne principale
    \node (start) [start] {Début};
    \node (a) [process, below of=start, yshift=-0.5cm] {Utilisateur sélectionne\\centre et type de logement};
    \node (b) [process, below of=a, yshift=-0.5cm] {Utilisateur choisit\\les dates};
    \node (c) [process, below of=b, yshift=-0.5cm] {Récupérer toutes les\\réservations existantes};
    \node (d) [process, below of=c, yshift=-0.5cm] {Filtrer les réservations\\par centre et type};
    \node (e) [process, below of=d, yshift=-0.5cm] {Vérifier les conflits\\de dates};
    \node (f) [decision, below of=e, yshift=-0.5cm] {Conflit détecté ?};
    \node (g) [process, below of=f, yshift=-0.5cm] {Calculer les créneaux\\disponibles};
    \node (h) [decision, below of=g, yshift=-0.5cm] {Y a-t-il des\\créneaux libres ?};
    \node (j) [process, below of=h, yshift=-0.5cm] {Retourner :\\Disponible avec créneaux};
    
    % Nœuds - Branches latérales
    \node (i) [process, left of=h, xshift=-3cm] {Retourner :\\Non disponible};
    \node (k) [process, right of=f, xshift=3cm] {Retourner :\\Totalement disponible};
    \node (end) [end, below of=i, yshift=-1cm] {Fin};
    
    % Flèches principales
    \draw [arrow] (start) -- (a);
    \draw [arrow] (a) -- (b);
    \draw [arrow] (b) -- (c);
    \draw [arrow] (c) -- (d);
    \draw [arrow] (d) -- (e);
    \draw [arrow] (e) -- (f);
    \draw [arrow] (f) -- node[right, yshift=0.2cm] {Oui} (g);
    \draw [arrow] (g) -- (h);
    \draw [arrow] (h) -- node[above, sloped] {Non} (i);
    \draw [arrow] (h) -- node[right, yshift=0.2cm] {Oui} (j);
    \draw [arrow] (f) -- node[above, sloped] {Non} (k);
    \draw [arrow] (i) -- (end);
    \draw [arrow] (j) -- (end);
    \draw [arrow] (k) -- (end);
    
\end{tikzpicture}
\caption{Processus de vérification de disponibilité}
\end{minipage}
\end{figure}

\section{Points de Contrôle et Validation}

\subsection{Validations Temporelles}
\begin{itemize}
    \item \textbf{Réservations} : Minimum 10 jours à l'avance
    \item \textbf{Paiement} : Maximum 24h après création
    \item \textbf{Codes SMS} : Expiration après 10 minutes
\end{itemize}

\subsection{Validations de Capacité}
\begin{itemize}
    \item \textbf{Logements} : Nombre de personnes ≤ capacité maximale
    \item \textbf{Centres} : Vérification de l'activité du centre
\end{itemize}

\subsection{Validations de Données}
\begin{itemize}
    \item \textbf{Unicité} : CIN, matricule, numéro de téléphone
    \item \textbf{Format} : Validation des formats de données
    \item \textbf{Cohérence} : Dates de début < dates de fin
\end{itemize}

\section{Gestion des Erreurs}

Chaque processus inclut des points de contrôle pour :
\begin{itemize}
    \item \textbf{Validation des données} : Vérification avant traitement
    \item \textbf{Gestion des conflits} : Résolution des conflits de disponibilité
    \item \textbf{Messages d'erreur} : Information claire à l'utilisateur
    \item \textbf{Rollback} : Annulation des opérations en cas d'erreur
\end{itemize}

\section{Tableau des Processus}

\begin{table}[h]
\centering
\begin{tabular}{|l|l|p{8cm}|}
\hline
\textbf{Processus} & \textbf{Complexité} & \textbf{Description} \\
\hline
Inscription Utilisateur & Élevée & Processus multi-étapes avec validations \\
\hline
Création Réservation & Élevée & Vérifications multiples et calculs \\
\hline
Paiement & Moyenne & Validation et confirmation \\
\hline
Vérification Disponibilité & Moyenne & Calculs de conflits temporels \\
\hline
Annulation & Faible & Mise à jour de statut simple \\
\hline
Envoi SMS & Faible & Appel API externe \\
\hline
\end{tabular}
\caption{Complexité des processus métier}
\end{table}

Ces diagrammes d'activité fournissent une vue détaillée des flux de processus du système COSONE, facilitant la compréhension et l'implémentation des fonctionnalités.

\end{document}
