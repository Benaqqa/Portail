\documentclass[12pt,a4paper]{article}
\usepackage[utf8]{inputenc}
\usepackage[french]{babel}
\usepackage{geometry}
\usepackage{graphicx}
\usepackage{hyperref}
\usepackage{listings}
\usepackage{xcolor}
\usepackage{enumitem}
\usepackage{amsmath}
\usepackage{amsfonts}
\usepackage{booktabs}
\usepackage{array}

\geometry{margin=2.5cm}

\lstset{
    language=properties,
    basicstyle=\ttfamily\small,
    backgroundcolor=\color{gray!10},
    frame=single,
    breaklines=true,
    showstringspaces=false
}

\title{\textbf{Documentation CMS - Projet COSONE}}
\author{Étudiant en Stage}
\date{\today}

\begin{document}

\maketitle

\section{Vue d'ensemble du CMS}

Le projet COSONE utilise \textbf{Spring Boot} comme framework principal avec \textbf{Thymeleaf} comme moteur de template pour la gestion de contenu et l'interface utilisateur.

\section{Architecture du CMS}

\subsection{Framework Principal : Spring Boot 3.5.4}

\textbf{Spring Boot} est utilisé comme CMS backend avec les caractéristiques suivantes :
\begin{itemize}
    \item \textbf{Version} : 3.5.4
    \item \textbf{Java} : Version 17
    \item \textbf{Architecture} : MVC (Model-View-Controller)
    \item \textbf{Injection de dépendances} : Automatique via annotations Spring
\end{itemize}

\subsection{Moteur de Template : Thymeleaf}

\textbf{Thymeleaf} est utilisé pour la génération des pages web :
\begin{itemize}
    \item \textbf{Intégration} : \texttt{spring-boot-starter-thymeleaf}
    \item \textbf{Encodage} : UTF-8
    \item \textbf{Cache} : Désactivé en développement
    \item \textbf{Templates} : Stockés dans \texttt{src/main/resources/templates/}
\end{itemize}

\subsection{Base de Données : PostgreSQL}

\textbf{PostgreSQL} est utilisé comme SGBD :
\begin{itemize}
    \item \textbf{Driver} : \texttt{org.postgresql:postgresql}
    \item \textbf{ORM} : JPA/Hibernate
    \item \textbf{Configuration} : \texttt{spring.jpa.hibernate.ddl-auto=update}
\end{itemize}

\subsection{Sécurité : Spring Security}

\textbf{Spring Security} gère l'authentification et l'autorisation :
\begin{itemize}
    \item \textbf{Authentification} : Basée sur les utilisateurs en base
    \item \textbf{Rôles} : USER, ADMIN
    \item \textbf{Protection} : CSRF activé
\end{itemize}

\section{Structure du CMS}

\subsection{Modèles de Données (Entités)}

\begin{enumerate}
    \item \textbf{User} - Gestion des utilisateurs
    \item \textbf{Reservation} - Gestion des réservations
    \item \textbf{Centre} - Centres de vacances
    \item \textbf{TypeLogement} - Types de logements disponibles
    \item \textbf{PersonneAccompagnement} - Personnes accompagnant les réservations
    \item \textbf{ExternAuthCode} - Codes d'authentification externe
    \item \textbf{PhoneVerificationCode} - Codes de vérification SMS
\end{enumerate}

\subsection{Contrôleurs (Controllers)}

\begin{enumerate}
    \item \textbf{HomeController} - Page d'accueil et landing page
    \item \textbf{AuthController} - Authentification et inscription
    \item \textbf{ReservationController} - Gestion des réservations
    \item \textbf{EspaceReservationController} - Espace personnel des utilisateurs
\end{enumerate}

\subsection{Services}

\begin{enumerate}
    \item \textbf{ReservationService} - Logique métier des réservations
    \item \textbf{EmailService} - Envoi d'emails
    \item \textbf{SmsService} - Envoi de SMS
    \item \textbf{WordPressService} - Intégration WordPress
    \item \textbf{HomeContentService} - Gestion du contenu de la page d'accueil
\end{enumerate}

\subsection{Repositories}

Interface avec JPA pour l'accès aux données :
\begin{itemize}
    \item \textbf{UserRepository}
    \item \textbf{ReservationRepository}
    \item \textbf{CentreRepository}
    \item \textbf{TypeLogementRepository}
    \item \textbf{ExternAuthCodeRepository}
    \item \textbf{PhoneVerificationCodeRepository}
\end{itemize}

\section{Fonctionnalités du CMS}

\subsection{Gestion des Utilisateurs}
\begin{itemize}
    \item Inscription avec code d'authentification externe
    \item Vérification par SMS
    \item Gestion des rôles (USER/ADMIN)
    \item Authentification sécurisée
\end{itemize}

\subsection{Gestion des Réservations}
\begin{itemize}
    \item Création de réservations
    \item Gestion des statuts (EN\_ATTENTE\_PAIEMENT, PAYEE, CONFIRMEE, etc.)
    \item Calcul automatique des prix
    \item Vérification de disponibilité
\end{itemize}

\subsection{Gestion des Centres}
\begin{itemize}
    \item CRUD des centres de vacances
    \item Import/Export CSV
    \item Gestion des types de logements
\end{itemize}

\subsection{Intégration WordPress}
\begin{itemize}
    \item Récupération d'articles depuis WordPress
    \item Affichage dynamique du contenu
    \item API REST WordPress
\end{itemize}

\section{Configuration}

\subsection{Fichier application.properties}

\begin{lstlisting}[caption=Configuration de l'application]
# Base de données
spring.datasource.url=jdbc:postgresql://localhost:5432/cosone_db
spring.datasource.username=postgres
spring.datasource.password=12345678

# JPA/Hibernate
spring.jpa.hibernate.ddl-auto=update
spring.jpa.show-sql=true
spring.jpa.properties.hibernate.dialect=org.hibernate.dialect.PostgreSQLDialect

# Thymeleaf
spring.thymeleaf.encoding=UTF-8
spring.thymeleaf.cache=false

# WordPress API
wordpress.api.url=https://moubarakmimo-barit.wordpress.com/
wordpress.api.timeout=5000
\end{lstlisting}

\section{Avantages de cette Architecture}

\begin{enumerate}
    \item \textbf{Scalabilité} : Spring Boot permet une montée en charge facile
    \item \textbf{Sécurité} : Spring Security offre une sécurité robuste
    \item \textbf{Maintenabilité} : Architecture MVC claire et modulaire
    \item \textbf{Flexibilité} : Thymeleaf permet des templates dynamiques
    \item \textbf{Performance} : JPA/Hibernate optimise les requêtes
    \item \textbf{Intégration} : Facile d'intégrer de nouveaux services
\end{enumerate}

\section{Technologies Utilisées}

\begin{itemize}
    \item \textbf{Backend} : Spring Boot 3.5.4, Spring Security, Spring Data JPA
    \item \textbf{Frontend} : Thymeleaf, HTML5, CSS3, JavaScript
    \item \textbf{Base de données} : PostgreSQL
    \item \textbf{Build} : Maven
    \item \textbf{Java} : Version 17
\end{itemize}

Cette architecture CMS moderne et robuste permet une gestion efficace du système de réservation COSONE avec une interface utilisateur intuitive et une sécurité renforcée.

\end{document}
