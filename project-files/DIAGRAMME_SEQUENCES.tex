\documentclass[12pt,a4paper]{article}
\usepackage[utf8]{inputenc}
\usepackage[french]{babel}
\usepackage{geometry}
\usepackage{graphicx}
\usepackage{hyperref}
\usepackage{listings}
\usepackage{xcolor}
\usepackage{enumitem}
\usepackage{amsmath}
\usepackage{amsfonts}
\usepackage{booktabs}
\usepackage{array}
\usepackage{tikz}
\usetikzlibrary{shapes,arrows,positioning,shadows,decorations.pathmorphing}

\geometry{margin=2.5cm}

\lstset{
    language=Java,
    basicstyle=\ttfamily\small,
    backgroundcolor=\color{gray!10},
    frame=single,
    breaklines=true,
    showstringspaces=false
}

\title{\textbf{Diagramme de Séquences - Projet COSONE}}
\author{Étudiant en Stage}
\date{\today}

\begin{document}

\maketitle

\section{Vue d'ensemble}

Ce document présente les diagrammes de séquences du système COSONE, décrivant les interactions entre les différents composants pour les principales fonctionnalités.

\section{Séquence d'Inscription Utilisateur}

\begin{figure}[h]
\centering
\begin{tikzpicture}[node distance=1.5cm, auto]
    % Style des nœuds
    \tikzstyle{actor} = [rectangle, draw, fill=blue!20, text width=2cm, text centered, minimum height=1cm, drop shadow]
    \tikzstyle{component} = [rectangle, draw, fill=green!20, text width=2.5cm, text centered, minimum height=1cm, drop shadow]
    \tikzstyle{arrow} = [thick,->,>=stealth]
    \tikzstyle{response} = [thick,<-,>=stealth, dashed]
    
    % Acteurs et composants
    \node (user) [actor] {Utilisateur};
    \node (browser) [component, right of=user, xshift=1cm] {Web Browser};
    \node (authController) [component, right of=browser, xshift=1cm] {AuthController};
    \node (externAuthRepo) [component, right of=authController, xshift=1cm] {ExternAuthCode\\Repository};
    \node (userRepo) [component, right of=externAuthRepo, xshift=1cm] {UserRepository};
    \node (smsService) [component, below of=userRepo, yshift=-1cm] {SmsService};
    \node (phoneVerifRepo) [component, left of=smsService, xshift=-1cm] {PhoneVerification\\Repository};
    
    % Interactions
    \draw [arrow] (user) -- node[above] {1. Accède à la page d'inscription} (browser);
    \draw [arrow] (browser) -- node[above] {2. GET /register} (authController);
    \draw [response] (authController) -- node[below] {3. Afficher formulaire} (browser);
    \draw [arrow] (browser) -- node[above] {4. POST /register (code, infos)} (authController);
    \draw [arrow] (authController) -- node[above] {5. findByCodeAndUsedFalse(code)} (externAuthRepo);
    \draw [response] (externAuthRepo) -- node[below] {6. Code trouvé} (authController);
    \draw [arrow] (authController) -- node[above] {7. marquerCodeCommeUtilise(code)} (externAuthRepo);
    \draw [arrow] (authController) -- node[above] {8. findByUsername(username)} (userRepo);
    \draw [response] (userRepo) -- node[below] {9. Utilisateur non trouvé} (authController);
    \draw [arrow] (authController) -- node[above] {10. save(nouvelUtilisateur)} (userRepo);
    \draw [response] (userRepo) -- node[below] {11. Utilisateur créé} (authController);
    \draw [arrow] (authController) -- node[above] {12. genererCodeVerification()} (smsService);
    \draw [response] (smsService) -- node[below] {13. Code généré} (authController);
    \draw [arrow] (authController) -- node[above] {14. save(codeVerification)} (phoneVerifRepo);
    \draw [response] (phoneVerifRepo) -- node[below] {15. Code sauvegardé} (authController);
    \draw [arrow] (authController) -- node[above] {16. envoyerSms(phone, code)} (smsService);
    \draw [response] (smsService) -- node[below] {17. SMS envoyé} (authController);
    \draw [arrow] (authController) -- node[above] {18. Redirection vers vérification SMS} (browser);
    \draw [response] (browser) -- node[below] {19. Afficher page de vérification} (user);
    
\end{tikzpicture}
\caption{Séquence d'inscription utilisateur}
\end{figure}

\section{Séquence de Création de Réservation}

\begin{figure}[h]
\centering
\begin{tikzpicture}[node distance=1.2cm, auto]
    % Style des nœuds
    \tikzstyle{actor} = [rectangle, draw, fill=blue!20, text width=2cm, text centered, minimum height=1cm, drop shadow]
    \tikzstyle{component} = [rectangle, draw, fill=green!20, text width=2.2cm, text centered, minimum height=1cm, drop shadow]
    \tikzstyle{arrow} = [thick,->,>=stealth]
    \tikzstyle{response} = [thick,<-,>=stealth, dashed]
    
    % Acteurs et composants
    \node (user) [actor] {Utilisateur};
    \node (browser) [component, right of=user, xshift=0.8cm] {Web Browser};
    \node (resController) [component, right of=browser, xshift=0.8cm] {Reservation\\Controller};
    \node (centreRepo) [component, right of=resController, xshift=0.8cm] {Centre\\Repository};
    \node (typeRepo) [component, right of=centreRepo, xshift=0.8cm] {TypeLogement\\Repository};
    \node (resService) [component, below of=resController, yshift=-1cm] {Reservation\\Service};
    \node (resRepo) [component, right of=resService, xshift=1.5cm] {Reservation\\Repository};
    
    % Interactions
    \draw [arrow] (user) -- node[above] {1. Accède à la page de réservation} (browser);
    \draw [arrow] (browser) -- node[above] {2. GET /reservation} (resController);
    \draw [arrow] (resController) -- node[above] {3. findByActifTrueOrderByNom()} (centreRepo);
    \draw [response] (centreRepo) -- node[below] {4. Liste des centres} (resController);
    \draw [arrow] (resController) -- node[above] {5. findByActifTrueOrderByNom()} (typeRepo);
    \draw [response] (typeRepo) -- node[below] {6. Liste des types} (resController);
    \draw [response] (resController) -- node[below] {7. Afficher formulaire} (browser);
    \draw [response] (browser) -- node[below] {8. Afficher centres et types} (user);
    \draw [arrow] (user) -- node[above] {9. Remplit le formulaire} (browser);
    \draw [arrow] (browser) -- node[above] {10. POST /reservation/creer} (resController);
    \draw [arrow] (resController) -- node[above] {11. findById(centreId)} (centreRepo);
    \draw [response] (centreRepo) -- node[below] {12. Centre trouvé} (resController);
    \draw [arrow] (resController) -- node[above] {13. findById(typeId)} (typeRepo);
    \draw [response] (typeRepo) -- node[below] {14. Type trouvé} (resController);
    \draw [arrow] (resController) -- node[above] {15. estDisponible(reservation)} (resService);
    \draw [arrow] (resService) -- node[above] {16. findByCentreAndType()} (resRepo);
    \draw [response] (resRepo) -- node[below] {17. Réservations existantes} (resService);
    \draw [response] (resService) -- node[below] {18. Disponible} (resController);
    \draw [arrow] (resController) -- node[above] {19. creerReservation()} (resService);
    \draw [arrow] (resService) -- node[above] {20. save(reservation)} (resRepo);
    \draw [response] (resRepo) -- node[below] {21. Réservation créée} (resService);
    \draw [response] (resService) -- node[below] {22. Réservation sauvegardée} (resController);
    \draw [arrow] (resController) -- node[above] {23. Redirection vers confirmation} (browser);
    \draw [response] (browser) -- node[below] {24. Afficher page de confirmation} (user);
    
\end{tikzpicture}
\caption{Séquence de création de réservation}
\end{figure}

\section{Séquence de Confirmation de Paiement}

\begin{figure}[h]
\centering
\begin{tikzpicture}[node distance=1.2cm, auto]
    % Style des nœuds
    \tikzstyle{actor} = [rectangle, draw, fill=blue!20, text width=2cm, text centered, minimum height=1cm, drop shadow]
    \tikzstyle{component} = [rectangle, draw, fill=green!20, text width=2.5cm, text centered, minimum height=1cm, drop shadow]
    \tikzstyle{arrow} = [thick,->,>=stealth]
    \tikzstyle{response} = [thick,<-,>=stealth, dashed]
    
    % Acteurs et composants
    \node (user) [actor] {Utilisateur};
    \node (browser) [component, right of=user, xshift=1cm] {Web Browser};
    \node (resController) [component, right of=browser, xshift=1cm] {Reservation\\Controller};
    \node (resRepo) [component, right of=resController, xshift=1cm] {Reservation\\Repository};
    \node (resService) [component, below of=resController, yshift=-1cm] {Reservation\\Service};
    \node (emailService) [component, right of=resService, xshift=1cm] {EmailService};
    
    % Interactions
    \draw [arrow] (user) -- node[above] {1. Accède à la confirmation} (browser);
    \draw [arrow] (browser) -- node[above] {2. GET /reservation/confirmation/{id}} (resController);
    \draw [arrow] (resController) -- node[above] {3. findById(id)} (resRepo);
    \draw [response] (resRepo) -- node[below] {4. Réservation trouvée} (resController);
    \draw [response] (resController) -- node[below] {5. Afficher détails et options} (browser);
    \draw [response] (browser) -- node[below] {6. Afficher formulaire de paiement} (user);
    \draw [arrow] (user) -- node[above] {7. Saisit méthode et référence} (browser);
    \draw [arrow] (browser) -- node[above] {8. POST /reservation/paiement/{id}} (resController);
    \draw [arrow] (resController) -- node[above] {9. confirmerPaiement(id, methode, ref)} (resService);
    \draw [arrow] (resService) -- node[above] {10. findById(id)} (resRepo);
    \draw [response] (resRepo) -- node[below] {11. Réservation trouvée} (resService);
    \draw [arrow] (resService) -- node[above] {12. updateStatutEtPaiement()} (resRepo);
    \draw [response] (resRepo) -- node[below] {13. Réservation mise à jour} (resService);
    \draw [response] (resService) -- node[below] {14. Paiement confirmé} (resController);
    \draw [arrow] (resController) -- node[above] {15. envoyerConfirmationPaiement()} (emailService);
    \draw [response] (emailService) -- node[below] {16. Email envoyé} (resController);
    \draw [arrow] (resController) -- node[above] {17. Redirection vers confirmation} (browser);
    \draw [response] (browser) -- node[below] {18. Afficher confirmation de paiement} (user);
    
\end{tikzpicture}
\caption{Séquence de confirmation de paiement}
\end{figure}

\section{Séquence de Connexion Utilisateur}

\begin{figure}[h]
\centering
\begin{tikzpicture}[node distance=1.5cm, auto]
    % Style des nœuds
    \tikzstyle{actor} = [rectangle, draw, fill=blue!20, text width=2cm, text centered, minimum height=1cm, drop shadow]
    \tikzstyle{component} = [rectangle, draw, fill=green!20, text width=2.5cm, text centered, minimum height=1cm, drop shadow]
    \tikzstyle{arrow} = [thick,->,>=stealth]
    \tikzstyle{response} = [thick,<-,>=stealth, dashed]
    
    % Acteurs et composants
    \node (user) [actor] {Utilisateur};
    \node (browser) [component, right of=user, xshift=1cm] {Web Browser};
    \node (authController) [component, right of=browser, xshift=1cm] {AuthController};
    \node (userRepo) [component, right of=authController, xshift=1cm] {UserRepository};
    \node (securityContext) [component, right of=userRepo, xshift=1cm] {SecurityContext};
    
    % Interactions
    \draw [arrow] (user) -- node[above] {1. Accède à la page de connexion} (browser);
    \draw [arrow] (browser) -- node[above] {2. GET /login} (authController);
    \draw [response] (authController) -- node[below] {3. Afficher formulaire} (browser);
    \draw [response] (browser) -- node[below] {4. Afficher formulaire} (user);
    \draw [arrow] (user) -- node[above] {5. Saisit username et password} (browser);
    \draw [arrow] (browser) -- node[above] {6. POST /login (username, password)} (authController);
    \draw [arrow] (authController) -- node[above] {7. findByUsername(username)} (userRepo);
    \draw [response] (userRepo) -- node[below] {8. Utilisateur trouvé} (authController);
    \draw [arrow] (authController) -- node[above] {9. verifierMotDePasse()} (authController);
    \draw [arrow] (authController) -- node[above] {10. setAuthentication(utilisateur)} (securityContext);
    \draw [response] (securityContext) -- node[below] {11. Utilisateur authentifié} (authController);
    \draw [arrow] (authController) -- node[above] {12. Redirection vers page d'accueil} (browser);
    \draw [response] (browser) -- node[below] {13. Afficher page d'accueil connecté} (user);
    
\end{tikzpicture}
\caption{Séquence de connexion utilisateur}
\end{figure}

\section{Séquence d'Envoi de SMS}

\begin{figure}[h]
\centering
\begin{tikzpicture}[node distance=1.5cm, auto]
    % Style des nœuds
    \tikzstyle{component} = [rectangle, draw, fill=green!20, text width=2.5cm, text centered, minimum height=1cm, drop shadow]
    \tikzstyle{external} = [rectangle, draw, fill=orange!20, text width=2.5cm, text centered, minimum height=1cm, drop shadow]
    \tikzstyle{arrow} = [thick,->,>=stealth]
    \tikzstyle{response} = [thick,<-,>=stealth, dashed]
    
    % Composants
    \node (smsService) [component] {SmsService};
    \node (phoneVerifRepo) [component, right of=smsService, xshift=2cm] {PhoneVerification\\Repository};
    \node (apiSms) [external, right of=phoneVerifRepo, xshift=2cm] {API SMS\\Externe};
    
    % Interactions
    \draw [arrow] (smsService) -- node[above] {1. genererCodeVerification()} (smsService);
    \draw [arrow] (smsService) -- node[above] {2. save(codeVerification)} (phoneVerifRepo);
    \draw [response] (phoneVerifRepo) -- node[below] {3. Code sauvegardé} (smsService);
    \draw [arrow] (smsService) -- node[above] {4. formaterMessageSMS()} (smsService);
    \draw [arrow] (smsService) -- node[above] {5. envoyerSMS(phone, message)} (apiSms);
    \draw [response] (apiSms) -- node[below] {6. Statut envoi} (smsService);
    \draw [arrow] (smsService) -- node[above] {7. loggerSucces() ou loggerErreur()} (smsService);
    
\end{tikzpicture}
\caption{Séquence d'envoi de SMS}
\end{figure}

\section{Points Clés des Interactions}

\subsection{Gestion des Erreurs}
Chaque interaction inclut des vérifications de validité :
\begin{itemize}
    \item Les erreurs sont propagées avec des messages explicites
    \item Les rollbacks sont effectués en cas d'échec
    \item Validation des données à chaque étape
\end{itemize}

\subsection{Sécurité}
\begin{itemize}
    \item Authentification requise pour les opérations sensibles
    \item Validation des données à chaque étape
    \item Gestion des sessions utilisateur
    \item Protection CSRF activée
\end{itemize}

\subsection{Performance}
\begin{itemize}
    \item Utilisation de repositories pour l'accès aux données
    \item Mise en cache des données fréquemment utilisées
    \item Optimisation des requêtes de base de données
    \item Chargement paresseux des relations
\end{itemize}

\subsection{Traçabilité}
\begin{itemize}
    \item Logging des opérations importantes
    \item Suivi des modifications d'état
    \item Historique des actions utilisateur
    \item Audit trail complet
\end{itemize}

\section{Tableau des Séquences}

\begin{table}[h]
\centering
\begin{tabular}{|l|l|p{6cm}|}
\hline
\textbf{Séquence} & \textbf{Complexité} & \textbf{Composants Impliqués} \\
\hline
Inscription Utilisateur & Élevée & AuthController, UserRepository, SmsService \\
\hline
Création Réservation & Élevée & ReservationController, ReservationService, Repositories \\
\hline
Confirmation Paiement & Moyenne & ReservationController, ReservationService, EmailService \\
\hline
Connexion Utilisateur & Faible & AuthController, UserRepository, SecurityContext \\
\hline
Envoi SMS & Faible & SmsService, API Externe \\
\hline
Vérification Disponibilité & Moyenne & ReservationService, Repositories \\
\hline
\end{tabular}
\caption{Complexité des séquences d'interaction}
\end{table}

\section{Patterns d'Interaction}

\subsection{Pattern MVC}
\begin{itemize}
    \item \textbf{Controller} : Gère les requêtes HTTP
    \item \textbf{Service} : Contient la logique métier
    \item \textbf{Repository} : Accède aux données
\end{itemize}

\subsection{Pattern Repository}
Abstraction de l'accès aux données avec méthodes de requête personnalisées.

\subsection{Pattern Service Layer}
Couche de service pour isoler la logique métier des contrôleurs.

\subsection{Pattern Dependency Injection}
Injection automatique des dépendances par Spring Framework.

Ces diagrammes de séquences fournissent une vue détaillée des interactions entre les composants du système COSONE, facilitant la compréhension du flux de données et des opérations.

\end{document}
