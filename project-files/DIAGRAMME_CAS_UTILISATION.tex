\documentclass[12pt,a4paper]{article}
\usepackage[utf8]{inputenc}
\usepackage[french]{babel}
\usepackage{geometry}
\usepackage{graphicx}
\usepackage{hyperref}
\usepackage{listings}
\usepackage{xcolor}
\usepackage{enumitem}
\usepackage{amsmath}
\usepackage{amsfonts}
\usepackage{booktabs}
\usepackage{array}
\usepackage{tikz}
\usetikzlibrary{shapes,arrows,positioning,shadows}

\geometry{margin=2.5cm}

\title{\textbf{Diagramme de Cas d'Utilisation - Projet COSONE}}
\author{Étudiant en Stage}
\date{\today}

\begin{document}

\maketitle

\section{Vue d'ensemble}

Ce diagramme présente les cas d'utilisation du système COSONE, montrant les interactions entre les acteurs et le système pour la gestion des réservations de centres de vacances.

\section{Diagramme de Cas d'Utilisation}

\begin{figure}[h]
\centering
\begin{tikzpicture}[node distance=1.5cm, auto]
    % Style des nœuds
    \tikzstyle{actor} = [ellipse, draw, fill=blue!20, text width=2.5cm, text centered, minimum height=1cm, drop shadow]
    \tikzstyle{usecase} = [ellipse, draw, fill=green!20, text width=3cm, text centered, minimum height=0.8cm, drop shadow]
    \tikzstyle{system} = [rectangle, draw, fill=gray!20, text width=15cm, text centered, minimum height=8cm, drop shadow]
    \tikzstyle{arrow} = [thick,->,>=stealth]
    \tikzstyle{include} = [thick,->,>=stealth, dashed, color=red]
    \tikzstyle{extend} = [thick,->,>=stealth, dashed, color=blue]
    
    % Acteurs
    \node (utilisateur) [actor] {\textbf{Utilisateur}\\\textit{Employé}};
    \node (admin) [actor, right of=utilisateur, xshift=3cm] {\textbf{Administrateur}\\\textit{Système}};
    \node (systeme) [actor, right of=admin, xshift=3cm] {\textbf{Système}\\\textit{Externe}};
    \node (wordpress) [actor, below of=systeme, yshift=-1cm] {\textbf{WordPress}\\\textit{CMS}};
    
    % Système
    \node (systemBox) [system, below of=utilisateur, yshift=-2cm] {};
    
    % Cas d'utilisation - Authentification
    \node (uc1) [usecase, below of=utilisateur, yshift=-3cm, xshift=-2cm] {Inscription avec\\code externe};
    \node (uc2) [usecase, below of=uc1, yshift=-1cm] {Vérification\\par SMS};
    \node (uc3) [usecase, below of=uc2, yshift=-1cm] {Connexion};
    \node (uc4) [usecase, below of=uc3, yshift=-1cm] {Création de\\mot de passe};
    \node (uc5) [usecase, below of=uc4, yshift=-1cm] {Déconnexion};
    
    % Cas d'utilisation - Réservations
    \node (uc6) [usecase, right of=uc1, xshift=2cm] {Consulter les\\centres disponibles};
    \node (uc7) [usecase, right of=uc2, xshift=2cm] {Consulter les\\types de logement};
    \node (uc8) [usecase, right of=uc3, xshift=2cm] {Vérifier la\\disponibilité};
    \node (uc9) [usecase, right of=uc4, xshift=2cm] {Créer une\\réservation};
    \node (uc10) [usecase, right of=uc5, xshift=2cm] {Calculer le prix};
    
    % Cas d'utilisation - Administration
    \node (uc18) [usecase, right of=uc6, xshift=2cm] {Gérer les\\utilisateurs};
    \node (uc19) [usecase, right of=uc7, xshift=2cm] {Générer des codes\\d'authentification};
    \node (uc20) [usecase, right of=uc8, xshift=2cm] {Gérer les\\centres};
    \node (uc21) [usecase, right of=uc9, xshift=2cm] {Gérer les types\\de logement};
    \node (uc22) [usecase, right of=uc10, xshift=2cm] {Consulter les\\statistiques};
    
    % Relations Utilisateur
    \draw [arrow] (utilisateur) -- (uc1);
    \draw [arrow] (utilisateur) -- (uc2);
    \draw [arrow] (utilisateur) -- (uc3);
    \draw [arrow] (utilisateur) -- (uc4);
    \draw [arrow] (utilisateur) -- (uc5);
    \draw [arrow] (utilisateur) -- (uc6);
    \draw [arrow] (utilisateur) -- (uc7);
    \draw [arrow] (utilisateur) -- (uc8);
    \draw [arrow] (utilisateur) -- (uc9);
    \draw [arrow] (utilisateur) -- (uc10);
    
    % Relations Administrateur
    \draw [arrow] (admin) -- (uc3);
    \draw [arrow] (admin) -- (uc5);
    \draw [arrow] (admin) -- (uc18);
    \draw [arrow] (admin) -- (uc19);
    \draw [arrow] (admin) -- (uc20);
    \draw [arrow] (admin) -- (uc21);
    \draw [arrow] (admin) -- (uc22);
    
    % Relations d'inclusion
    \draw [include] (uc9) -- (uc8);
    \draw [include] (uc9) -- (uc10);
    
    % Relations d'extension
    \draw [extend] (uc2) -- (uc1);
    \draw [extend] (uc4) -- (uc2);
    
    % Légende
    \node [below of=systemBox, yshift=-1cm] {\textbf{Légende :} \textcolor{red}{---} Inclusion, \textcolor{blue}{---} Extension};
    
\end{tikzpicture}
\caption{Diagramme de cas d'utilisation du système COSONE}
\end{figure}

\section{Description Détaillée des Cas d'Utilisation}

\subsection{Authentification}

\subsubsection{UC1 - Inscription avec code externe}
\begin{itemize}
    \item \textbf{Acteur principal} : Utilisateur
    \item \textbf{Préconditions} : Avoir un code d'authentification valide
    \item \textbf{Scénario principal} :
    \begin{enumerate}
        \item L'utilisateur saisit son code d'authentification
        \item Le système vérifie la validité du code
        \item L'utilisateur saisit ses informations personnelles
        \item Le système crée le compte utilisateur
        \item Le système envoie un SMS de vérification
    \end{enumerate}
\end{itemize}

\subsubsection{UC2 - Vérification par SMS}
\begin{itemize}
    \item \textbf{Acteur principal} : Utilisateur
    \item \textbf{Préconditions} : Avoir reçu un SMS de vérification
    \item \textbf{Scénario principal} :
    \begin{enumerate}
        \item L'utilisateur reçoit un SMS avec un code
        \item L'utilisateur saisit le code reçu
        \item Le système vérifie le code
        \item Le système valide le numéro de téléphone
    \end{enumerate}
\end{itemize}

\subsubsection{UC3 - Connexion}
\begin{itemize}
    \item \textbf{Acteur principal} : Utilisateur, Administrateur
    \item \textbf{Préconditions} : Avoir un compte valide
    \item \textbf{Scénario principal} :
    \begin{enumerate}
        \item L'utilisateur saisit ses identifiants
        \item Le système vérifie les informations
        \item Le système authentifie l'utilisateur
        \item L'utilisateur accède à son espace
    \end{enumerate}
\end{itemize}

\subsection{Gestion des Réservations}

\subsubsection{UC6 - Consulter les centres disponibles}
\begin{itemize}
    \item \textbf{Acteur principal} : Utilisateur
    \item \textbf{Scénario principal} :
    \begin{enumerate}
        \item L'utilisateur accède à la page des centres
        \item Le système affiche la liste des centres actifs
        \item L'utilisateur peut filtrer par ville ou critères
    \end{enumerate}
\end{itemize}

\subsubsection{UC9 - Créer une réservation}
\begin{itemize}
    \item \textbf{Acteur principal} : Utilisateur
    \item \textbf{Préconditions} : Être connecté
    \item \textbf{Scénario principal} :
    \begin{enumerate}
        \item L'utilisateur sélectionne un centre et un type de logement
        \item L'utilisateur choisit les dates de séjour
        \item Le système vérifie la disponibilité
        \item L'utilisateur saisit le nombre de personnes
        \item L'utilisateur ajoute les personnes d'accompagnement
        \item Le système calcule le prix total
        \item L'utilisateur confirme la réservation
        \item Le système crée la réservation avec statut "EN\_ATTENTE\_PAIEMENT"
    \end{enumerate}
\end{itemize}

\subsubsection{UC13 - Confirmer le paiement}
\begin{itemize}
    \item \textbf{Acteur principal} : Utilisateur
    \item \textbf{Préconditions} : Avoir une réservation en attente de paiement
    \item \textbf{Scénario principal} :
    \begin{enumerate}
        \item L'utilisateur accède à sa réservation
        \item L'utilisateur sélectionne la méthode de paiement
        \item L'utilisateur saisit la référence de paiement
        \item Le système met à jour le statut de la réservation
        \item Le système envoie une confirmation par email
    \end{enumerate}
\end{itemize}

\subsection{Administration}

\subsubsection{UC18 - Gérer les utilisateurs}
\begin{itemize}
    \item \textbf{Acteur principal} : Administrateur
    \item \textbf{Scénario principal} :
    \begin{enumerate}
        \item L'administrateur accède à la liste des utilisateurs
        \item L'administrateur peut consulter les informations
        \item L'administrateur peut modifier les rôles
        \item L'administrateur peut désactiver des comptes
    \end{enumerate}
\end{itemize}

\subsubsection{UC19 - Générer des codes d'authentification}
\begin{itemize}
    \item \textbf{Acteur principal} : Administrateur
    \item \textbf{Scénario principal} :
    \begin{enumerate}
        \item L'administrateur accède à la génération de codes
        \item L'administrateur saisit les informations de l'employé
        \item Le système génère un code unique
        \item Le système sauvegarde le code en base
    \end{enumerate}
\end{itemize}

\section{Relations entre Cas d'Utilisation}

\subsection{Relations d'Inclusion}
\begin{itemize}
    \item \textbf{UC9 inclut UC8} : Créer une réservation inclut la vérification de disponibilité
    \item \textbf{UC9 inclut UC10} : Créer une réservation inclut le calcul du prix
    \item \textbf{UC9 inclut UC15} : Créer une réservation inclut l'ajout d'accompagnants
    \item \textbf{UC13 inclut UC10} : Confirmer le paiement inclut le calcul du prix
\end{itemize}

\subsection{Relations d'Extension}
\begin{itemize}
    \item \textbf{UC2 étend UC1} : La vérification SMS étend l'inscription
    \item \textbf{UC4 étend UC2} : La création de mot de passe étend la vérification SMS
    \item \textbf{UC12 étend UC11} : Annuler une réservation étend la consultation des réservations
    \item \textbf{UC14 étend UC11} : Consulter les détails étend la consultation des réservations
\end{itemize}

\section{Acteurs du Système}

\subsection{Utilisateur (Employé)}
\begin{itemize}
    \item \textbf{Description} : Employé de l'entreprise utilisant le système pour réserver des centres de vacances
    \item \textbf{Responsabilités} : S'inscrire, se connecter, créer des réservations, gérer ses accompagnants
\end{itemize}

\subsection{Administrateur}
\begin{itemize}
    \item \textbf{Description} : Personne responsable de la gestion du système et des utilisateurs
    \item \textbf{Responsabilités} : Gérer les utilisateurs, générer des codes, administrer les centres
\end{itemize}

\subsection{Système Externe}
\begin{itemize}
    \item \textbf{Description} : Services externes pour l'envoi d'emails et SMS
    \item \textbf{Responsabilités} : Envoyer des notifications automatiques
\end{itemize}

\subsection{WordPress CMS}
\begin{itemize}
    \item \textbf{Description} : Système de gestion de contenu pour les articles et informations
    \item \textbf{Responsabilités} : Fournir du contenu dynamique à la page d'accueil
\end{itemize}

\section{Contraintes et Règles Métier}

\begin{enumerate}
    \item \textbf{Authentification} : Un code d'authentification ne peut être utilisé qu'une seule fois
    \item \textbf{Réservations} : Les réservations doivent être effectuées au moins 10 jours à l'avance
    \item \textbf{Paiement} : Le paiement doit être effectué dans les 24h suivant la création de la réservation
    \item \textbf{Accompagnants} : Le nombre total de personnes ne peut dépasser la capacité du logement
    \item \textbf{Disponibilité} : Vérification automatique des conflits de dates
\end{enumerate}

\section{Tableau Récapitulatif des Cas d'Utilisation}

\begin{table}[h]
\centering
\begin{tabular}{|l|l|l|p{6cm}|}
\hline
\textbf{ID} & \textbf{Nom} & \textbf{Acteur} & \textbf{Description} \\
\hline
UC1 & Inscription avec code externe & Utilisateur & Inscription avec validation par code \\
\hline
UC2 & Vérification par SMS & Utilisateur & Validation du numéro de téléphone \\
\hline
UC3 & Connexion & Utilisateur, Admin & Authentification dans le système \\
\hline
UC6 & Consulter les centres & Utilisateur & Visualisation des centres disponibles \\
\hline
UC9 & Créer une réservation & Utilisateur & Processus complet de réservation \\
\hline
UC13 & Confirmer le paiement & Utilisateur & Validation du paiement \\
\hline
UC18 & Gérer les utilisateurs & Administrateur & Administration des comptes \\
\hline
UC19 & Générer des codes & Administrateur & Création de codes d'authentification \\
\hline
\end{tabular}
\caption{Récapitulatif des principaux cas d'utilisation}
\end{table}

Ce diagramme de cas d'utilisation fournit une vue d'ensemble complète des fonctionnalités du système COSONE et des interactions entre les différents acteurs.

\end{document}
